% !TEX encoding = UTF-8 Unicode

\documentclass[a4j,10pt]{ltjsarticle}
\usepackage{graphicx}

\begin{document}

\title{待ち行列問題シミュレータの開発}
\author{1932008 伊豆原嵩章 指導教員 須田 宇宙 准教授}
\maketitle


\section{研究の概要}
\subsection{背景}
\begin{itemize}
\item {病院や店舗などでは,来場者の密度や窓口数が,回転率や人件費など経営に直接的に関わってくる}
\item {回転率を計算する手法としてモンテカルロ法を用いた待ち行列問題が存在する}
\end{itemize}

\subsection{問題点}
\begin{itemize}
\item{モンテカルロ法を学習する上で,計算結果だけ提示しても状況を理解しづらい}
\item{先行研究で開発された待ち行列問題シミュレータ教材が存在するが,ブラウザ上で動作しない}
\end{itemize}

\subsection{目的}
\begin{itemize}
\item{結果や状況を理解しやすくするために,アニメーションを用いて視覚的・直感的に理解できる待ち行列問題シミュレータ教材を開発する}
\end{itemize}

\section{何を研究成果とするか}開発物,最低限アニメーションを用いる

\section{前回の指摘}
\begin{itemize}
\item{アニメーションの実装はまだか?>計算部分のプログラムが完成してから取り掛かります}
\item{プログラムの書き方について>クラスを実装しました}
\end{itemize}

\section{進捗}
\begin{itemize}
\item{クラスを用いてプログラムを管理しやすくした}
\item{待ち時間を計算するプログラムを書いた}
\item{https://github.com/izuhara2/sotuken}
\end{itemize}

\section{スケジュール}
\begin{center}
 \begin{tabular}{lrr}
   〜11月13日(45週) &待ち時間計算プログラム試作\\
   〜11月20日(46週) &待ち時間計算プログラム試作・完成 \\
   〜11月27日(47週)&  グラフ・アニメーション制作
 \end{tabular}
\end{center}

\end{document}